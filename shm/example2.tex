% Writing second example
\begin{example}
A particle of mass m is executing simple harmonic motion along x axis under the 
action of a force $F = -kx$ with a period of $16$ sec. In the course of motion 
it crosses the equilibrium position at $t = 2 sec$ and acquire a velocity of $4 
m/sec$ at $t = 4 sec$. Find the equation of motion and the amplitude of 
oscillation. 
\end{example}
\solution{
\begin{sagesilent}
var('x,t,a,pi,T')
o = var('omega')
z = x == a*sin(o*(t-2))
z1 = a*sin(o*(t-2))
assume(a>0)
assume(o>0)
z2 = diff(z1,t)
z3 = 4 == z2
z4 = a == 4/o*cos(2*o)
z5 = o == 2*pi/T
z6 = 2*pi/16
z7 = pi/8
z8 = a == 4/(z7*1/sqrt(2))
a = 4/(z7*1/sqrt(2))
o = pi/8
z9 =x == a*sin(o*(t-2))
\end{sagesilent}
Let the equation be $\sage{z}$ where $a$= amplitude of motion.
Now \begin{equation} \frac{\mathrm{d}x}{\mathrm{d}t} = \sage{z2} \end{equation}

\begin{equation} \sage{z3} \end{equation}

\begin{equation} \sage{z4} \end{equation}

\begin{equation} \sage{z5} = \sage{z6} = \sage{z7} \end{equation}

\begin{equation} \therefore \sage{z8} \end{equation}

\begin{equation} \sage{z9} \end{equation}
}
