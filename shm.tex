%TITLE
\documentclass{article}
\title{Solving Differential Equations representing Simple Harmonic Motion}
\author{Amritpal Singh}
\usepackage{sagetex}

%ADDING HEADER AND FOOTER
\usepackage{geometry}
\usepackage{graphicx}	
\geometry{a4paper, margin=1.5in, top=1in, bottom=1in}

%MAIN SOURCE
\begin{document}
\maketitle

\begin{sagesilent}
#Declaring spring constant, mass, time as k, m, t respectively in SageMath
k=var('k')
m=var('m')
t=var('t')
x=function('x',t)

#Assuming spring constant and mass greater than zero
assume(k>0)
assume(m>0)

#Writing our differential equation in SageMath
de = diff(x,t,2)+(k*x)/m

#Soloving our differential equation in SageMath
z=desolve(de,dvar=x,ivar=t)
\end{sagesilent}

Consider $m$ be the mass of object, $k$ be spring constant, $x$ be a displacement from equilibrium state of a spring and $t$ is time. 

Therefore, the given differential equation of Simple Harmonic Motion is  

%PRINTING OUR DIFFERENTIAL EQUATION
\[
  \frac{\mathrm{d}^{2}x}{\mathrm{d}t^{2}} +  \sage{k*x/m} = 0.
\]

Solving the above differential equation, we get,

%PRINTING SOLUTION OF OUR DIFFERNTIAL EQUATION WHICH IS SOLOVED BY SAGEMATH
$$x = \sage{z}$$

\begin{sagesilent}
#Declaring mass and spring constant equal to 1
m=1
k=1
de = diff(x,t,2)+k*x/m

#Soloving our differential equation by setting up the initial or boundary conditions.
z2 = desolve(de,dvar=x,ivar=t,ics=[0,1,0])
#Plotting the graph of solution of our differential equation
pl1 = plot(z2,(t,0,10),color=('red'), figsize = (4,2.5),axes_labels=['$t$','$x$'], fontsize=7)

m=1
k=1
z3 = desolve(de,dvar=x,ivar=t,ics=[1,1,0])
print z3
pl2 = plot(z3,(t,0,10), color=('green'))

m=1
k=1
z4 = desolve(de,dvar=x,ivar=t, ics=[0,0,1])
pl3 = plot(z4,(t,0,10),color=('yellow'))

m=1
k=1
z5 = desolve(de,dvar=x,ivar=t, ics=[1,1,1])
pl4 = plot(z5,(t,0,10),color=('violet'))
\end{sagesilent}

When $m=1$, $k=1$, then the graph is

%PRINTING GRAPH BY USING SAGEMATH
$$\sageplot{pl1+pl2+pl3+pl4}$$

where red, green, yellow and violet curves are drawn when the initial or boundary conditions are [0,1,0], [1,1,0], [0,0,1] and [1,1,1] respectively.

The initial or boundary conditions means for a second-order equation, specify the initial $x$, $y$, and $\frac{\mathrm{d}x}{\mathrm{d}t}$, i.e. write [$t$, $x(t)$, $\frac{\mathrm{d}x}{\mathrm{d}t}$]

\end{document}
