\documentclass{report}

% PACKAGE USED
\usepackage{sagetex}
\usepackage{fancyheadings}
\pagestyle{fancy}

% ADDING HEADER AND FOOTER
\usepackage{geometry}
\usepackage{graphicx}	
\geometry{a4paper, margin=1.5in, top=1.5in, bottom=1in}

% TITLE PAGE
\title{Solving Differential Equations representing Simple Harmonic Motion}
\author{Amritpal Singh}

\begin{document}
\begin{titlepage}
\maketitle

% ADDING TABLE OG CONTENT
\tableofcontents
\end{titlepage}

% DECLARING CHAPTER NAME
\chapter{Simple Harmonic Motion}
\section{Derivation}

\begin{sagesilent}
# Declaring spring constant, mass, time as k, m, t respectively in SageMath
k=var('k')
m=var('m')
t=var('t')
x=function('x',t)

# Assuming spring constant and mass greater than zero
assume(k>0)
assume(m>0)

# Writing our differential equation in SageMath
de = diff(x,t,2)+(k*x)/m

# Soloving our differential equation in SageMath
z=desolve(de,dvar=x,ivar=t)
\end{sagesilent}

Consider $m$ be the mass of object, $k$ be spring constant, $x$ be a displacement from equilibrium state of a spring and $t$ is time.\\
Therefore, the given differential equation of Simple Harmonic Motion is  

% PRINTING OUR DIFFERENTIAL EQUATION
\[
  \frac{\mathrm{d}^{2}x}{\mathrm{d}t^{2}} +  \sage{k*x/m} = 0.
\]\\
Solving the above differential equation, we get,
% PRINTING A SOLUTION OF OUR DIFFERNTIAL EQUATION WHICH IS SOLOVED BY SAGEMATH
$$x = \sage{z}$$
\begin{sagesilent}
# Declaring mass and spring constant equal to 1
m=1
k=1
de = diff(x,t,2)+k*x/m

# Soloving our differential equation by setting up the initial or boundary conditions.
z2 = desolve(de,dvar=x,ivar=t,ics=[0,1,0])
# Plotting the graph of solution of our differential equation
pl1 = plot(z2,(t,0,10),color=('red'), figsize = (4,2.5),axes_labels=['$t$','$x$'], fontsize=7)

m=1
k=1
z3 = desolve(de,dvar=x,ivar=t,ics=[1,1,0])
print z3
pl2 = plot(z3,(t,0,10), color=('green'))

m=1
k=1
z4 = desolve(de,dvar=x,ivar=t, ics=[0,0,1])
pl3 = plot(z4,(t,0,10),color=('yellow'))

m=1
k=1
z5 = desolve(de,dvar=x,ivar=t, ics=[1,1,1])
pl4 = plot(z5,(t,0,10),color=('violet'))
\end{sagesilent}\\
When $m=1$, $k=1$, then the graph is
% ADDING CAPTION OF GRAPH
\begin{figure}[h!]
% PRINTING GRAPH BY USING SAGEMATH
$$\sageplot{pl1+pl2+pl3+pl4}$$
\caption{Graph}
\end{figure}\\
here red, green, yellow and violet curves are drawn when the initial or boundary conditions are [0,1,0], [1,1,0], [0,0,1] and [1,1,1] respectively.\\
The initial or boundary conditions means for a second-order differential equation, specify the initial $x$, $y$, and $\frac{\mathrm{d}x}{\mathrm{d}t}$, i.e. write [$t$, $x(t)$, $\frac{\mathrm{d}x}{\mathrm{d}t}$]

% NEXT SECTION OF OUR CHAPTER
\section{Numericals}

\textbf{Example:}\\
\textit{A spring at rest is suspened from the ceiling wthout mass. A 2 kg weight is then attached to this spring, stretching it 9.8 cm. From a position 2/3 m above equilibrium the weigth is give a downward velocity of 5 m/s. \\
(a) Find the equation of motion.\\
(b) What is the amplitude?\\
(c) At what times the mass first equilibrium?}\\ \\
\begin{sagesilent}
m=2
g=9.8
x=.098

# Calcuating spring constant
k=m*g/x
\end{sagesilent}
\textbf{Sol:}
Note $m = 2 kg$, $x =9.8 cm$, and $k = mg/x$ = $\sage{k.n(digits=4)}$.
\begin{sagesilent}
var('t')
x=function('x',t)
var('m')
var('k')
k=200
m=2
de =diff(x,t,2)+k*x/m

# Solving our differential equation
z = desolve(de,dvar= x,ivar=t)
\end{sagesilent}
Therefore, the general solution $x = \sage{z}$. Then by computing the above equation from the initial conditions $x(0) = -2/3$ (down is positive, up is negative), $x'(0) = 5$ we get,

\begin{sagesilent}
var('t')
x=function('x',t)
var('m')
var('k')
assume(k>0)
assume(m>0)
de =diff(x,t,2)+k*x/m
z = desolve(de,dvar= x,ivar=t)

m=2
k=200
de = diff(x,t,2)+k*x/m
# Solving our differential equation by putting the values of mass and spring constant and setting up initial conditions
z = desolve(de,dvar=x,ivar=t,ics=[0,-2/3,5])
pl = plot(z,(t,0,2),figsize=(4,2.5),fontsize = 7,axes_labels=['$t$','$x$'],color=('red')) 
\end{sagesilent}

$$x = \sage{z}$$\\
Now we write this in the more compact and useful form 
\begin{sagesilent}
o=var('omega')
p=var('phi')
var('t')
var('A,K1,K2')
e=A*sin(o*t+p)
f=K2*cos(o*t)+K1*sin(o*t)
a=sqrt((K1)^2+(K2)^2)
\end{sagesilent}
$$x = \sage{e} = \sage{f}$$\\
where $A = \sage{a}$ denotes the $amplitude$
\begin{sagesilent}
k1=-2/3
k2=1/2
a = sqrt(k1^2+k2^2)
\end{sagesilent}
$$A = \sage{a}$$
% ADDING CAPTION OF GRAPH
\begin{figure}[h]
% PLOTTING OUR GRAPH
$$\sageplot{pl}$$
\caption{Graph}

\end{figure}
\end{document}
