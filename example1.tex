% Creating new section
\section{Numericals}

% Writing first example
\begin{example}
A spring at rest is suspened from the ceiling wthout mass. A 2 kg weight is 
then attached to this spring, stretching it 9.8 cm. From a position 2/3 m above 
equilibrium the weigth is give a downward velocity of 5 m/s.
Find the equation of motion.
What is the amplitude?
At what times the mass first equilibrium?
\end{example}

\solution{
\begin{sagesilent}
# Initialising values of m, g, x
m=2
g=9.8
x=.098

# Calcuating spring constant
k=m*g/x
\end{sagesilent}
Note $m = 2 kg$, $x =9.8 cm$, and $k = mg/x$ = $\sage{k.n(digits=4)}$.
\begin{sagesilent}
var('t')
x=function('x',t)
var('m')
var('k')
k=200
m=2

# Writing a differential equation
de =diff(x,t,2)+k*x/m

# Solving a differential equation
z = desolve(de,dvar= x,ivar=t)
\end{sagesilent}
Therefore, the general solution $x = \sage{z}$. Then by computing the above 
equation from the initial conditions $x(0) = -2/3$ (down is positive, up is 
negative), $x'(0) = 5$ we get,

\begin{sagesilent}
var('t')
x=function('x',t)
var('m')
var('k')
assume(k>0)
assume(m>0)
de =diff(x,t,2)+k*x/m
z = desolve(de,dvar= x,ivar=t)

m=2
k=200
de = diff(x,t,2)+k*x/m
# Solving our differential equation by putting the values of mass and spring# constant and setting up initial conditions
z = desolve(de,dvar=x,ivar=t,ics=[0,-2/3,5])
pl = plot(z,(t,0,2),figsize=(4,2.5),fontsize = 
7,axes_labels=['$t$','$x$'],color=('red')) 
\end{sagesilent}

$$x = \sage{z}$$\\
Now we write this in the more compact and useful form 
\begin{sagesilent}
o=var('omega')
p=var('phi')
var('t')
var('A,K1,K2')
e=A*sin(o*t+p)
f=K2*cos(o*t)+K1*sin(o*t)
a=sqrt((K1)^2+(K2)^2)
\end{sagesilent}
$$x = \sage{e} = \sage{f}$$\\
where $A = \sage{a}$ denotes the $amplitude$
\begin{sagesilent}
k1=-2/3
k2=1/2
a = sqrt(k1^2+k2^2)
\end{sagesilent}
$$A = \sage{a}$$

\begin{figure}[h]
% Plotting graph
$$\sageplot{pl}$$
% Adding caption to graph
\caption{Displacement Time graph}
\end{figure}
}
